\documentclass[aps,nofootinbib]{revtex4}

\usepackage{mathpazo}

\usepackage{amsmath}
\usepackage{amsfonts,amssymb,amsthm, bbm,braket}
\usepackage{graphicx}   % need for figures
\usepackage{subfigure}  % and subfigures
\usepackage{amsbsy} % for bold greek letters
\usepackage[bold]{hhtensor}

%--------------------------------MACROS-------------------------------------------------------------------------------

\newcommand\red{\color{red}}
\newcommand\magn{\color{magenta}}
\newcommand\Tr{\mathrm{Tr}}

%--------------------------------ENVIRONMENTS-------------------------------------------------------------------------------

\newtheorem{definition}{Definition}
\newtheorem{corollary}{Corollary}
\newtheorem{theorem}{Theorem}
\newtheorem{lemma}{Lemma}
\newtheorem{proposition}{Proposition}
\newtheorem{question}{Question}

%--------------------------------ENVIRONMENTS-------------------------------------------------------------------------------

\usepackage[breaklinks=true]{hyperref}

%getting rid of hyperref's ugly boxes.
%From:http://tex.stackexchange.com/a/51349
\hypersetup{
  colorlinks   = true, %Colours links instead of ugly boxes
  urlcolor     = blue, %Colour for external hyperlinks
  linkcolor    = blue, %Colour of internal links
  citecolor   = magenta %Colour of citations
}

\begin{document}

\title{A Poor Man's Guide to Lie Theory}


\author{Ninnat Dangniam}
\affiliation{
Center for Quantum Information and Control,
University of New Mexico,
Albuquerque, New Mexico, 87131-0001}


%\author{Christopher Ferrie}
%\affiliation{Centre for Engineered Quantum Systems, School of Physics, The University of Sydney, Sydney, NSW 2006, Australia
%}

\date{\today}


\begin{abstract}
The goal of this guide is to give a feel about how Lie groups, Lie
algebras, and their representations interact as quickly as possible. It
is not meant to be very precise.
\end{abstract}

\maketitle

\section{Lie groups and Lie algebras}

The relation between Lie groups and Lie algebras is encapsulated in
this statement: \emph{the categories of connected, simply-connected
Lie groups and finite-dimensional Lie algebras are the same.} (\ref{Lie subgroup},\ref{adjoint},\ref{Lie algebra}) It cannot
be the category of all Lie groups because Lie algebras cannot know
the global topology of Lie groups. So we often study connected, simply-connected
covering groups instead of the groups that are of interest to us.

\begin{enumerate}
\item A \textbf{Lie group} $G$ is a group that is also a smooth manifold.
Roughly speaking, a \textbf{manifold} $M$ is a space that is locally
Euclidean i.e. the chart $\varphi$ mapping an open set $U$ to a
subset of $\mathbb{R}^{n}$ gives a point $p\in M$ coordinates in
$\mathbb{R}^{n}$. The \textbf{dimension} of a Lie group is $n$,
the dimension of the manifold.

\item \label{Lie algebra} A \textbf{Lie algebra} is a vector space $\mathfrak{g}$ over a field $k$, equipped with a bilinear map $\left[,\right]:\mathfrak{g}\times\mathfrak{g}\to\mathfrak{g}$,
the \textbf{Lie bracket}, which is skew-symmetric $\left[x,y\right]=-\left[y,x\right]$
(or equivalently, $\left[x,x\right]=0$) and satisfies the Jacobi
identity
\begin{align*}
\left[x,\left[y,z\right]\right]+\left[y,\left[z,x\right]\right]+\left[z,\left[x,y\right]\right] & =0.
\end{align*}
To understand the Jacobi identity, let us look at the \textbf{adjoint
map} that takes an element $x\in\mathfrak{g}$ and turns it into an
endomorphism (homomorphism to $\mathfrak{g}$ itself) of $\mathfrak{g}$.
\begin{align*}
\mbox{ad}:\mathfrak{g} & \to\mbox{End}\mathfrak{g}\\
\mbox{ad}x\left(y\right) & =\left[x,y\right].
\end{align*}
One can check that it really is a homomorphism:
\begin{align*}
\mbox{ad}\left(\left[x,y\right]\right) & =\left[\mbox{ad}x,\mbox{ad}y\right].
\end{align*}
So it is a representation of the Lie algebra called an \textbf{adjoint
representation }of $\mathfrak{g}$. It will play an important
role in the structure theory of semisimple Lie algebras.

Fixing the element $x$, the Jacobi idensity precisely means that
$\mbox{ad}x$ is a \textbf{derivation} (obeying Leibniz rule):
\begin{align*}
\mbox{ad}x\left(\left[y,z\right]\right) & =\left[\mbox{ad}x\left(y\right),z\right]+\left[y,\mbox{ad}x\left(z\right)\right].
\end{align*}
This ties in with another definition of a Lie algebra of a Lie group
as the tangent space at the identity $T_{e}G=\left\{ X|e^{tX}\in G,t\in\mathbb{R}\right\} $
of $G$, which is isomorphic to the set of partial differential operators
at the identity. Thus, the \textbf{dimension }of the Lie group and
that of its Lie algebra are the same.

Every Lie group has a Lie algebra. The best converse of this statement
is \textbf{Lie's third theorem}: any abstractly defined Lie algebra
has a corresponding connected, simply-connected Lie group.

\item \label{left-invariant vector field} The standard mathematical definition of the Lie algebra of a Lie group $G$ is the set of left-invariant vector fields on $G$. The
set of all vector fields can be given the structure of a Lie algebra
by the Lie bracket, but only left-invariant vector fields are determined
by their values at the identity, hence the definition in \ref{Lie algebra}. We prove
this statement here.

To prepare ourselves, think of a group element $g\in G$ as an automorphism
of the algebra $C^{\infty}(M)$ of smooth functions on a manifold $M$ that translates functions:
\begin{align*}
\left(gf\right)\left(p\right) & =f\left(g^{-1}p\right).
\end{align*}
This induces an action on vector fields. Define the \textbf{Adjoint
map }(with a capital ``A'') as ($X$ is a vector field)
\begin{align*}
\mbox{Ad}_{g}X & =gXg^{-1}.
\end{align*}
The value of $\mbox{Ad}_{g}X$ at point $q=gp$ can be computed at
point $p$.
\begin{align*}
\left(\mbox{Ad}_{g}X\right)_{q} & =dg_{p}\left(X_{p}\right).
\end{align*}
Let us unpack this little formula. $g:M\to M$ is an automorphism
of $M$. $dg_{p}:T_{p}M\to T_{gp}M=T_{q}M$ is the differential of
the map $g$ at point $p$. So both the right hand side $dg_{p}\left(X_{p}\right)$
and the left hand side are in the tangent space at $q$.

Now we can understand what left-invariant vector fields mean. Take $M=G$ the Lie group itself. Let $L_{g}$ be the left action on the group itself: $L_{g}\left(h\right)=gh$.
A vector field is \textbf{left-invariant }if $L_{g}X=XL_{g}$ i.e.
$\mbox{Ad}_{g}X=X$. As promised, the left-invariant vector field
at any point $g\in G$ can be seen to be determined by its value at the origin $X_{e}$:
\begin{align*}
X_{g} & =\left(\mbox{Ad}_{g}X\right)_{g}=dg_{e}\left(X_{e}\right)
\end{align*}


\item \label{adjoint} The Adjoint map $\mbox{Ad}_{g}:G\to\mbox{End}\mathfrak{g}$ (\ref{left-invariant vector field}) is a homomorphism
\begin{align*}
\mbox{Ad}_{g}\mbox{Ad}_{h} & =\mbox{Ad}_{gh}.
\end{align*}
So it is a representation of the group called the \textbf{Adjoint
representation} of $G$. Its differential is the adjoint map $\mbox{ad}:\mathfrak{g}\to\mbox{End}\mathfrak{g}$
from \ref{Lie algebra}. They are related by
\begin{align*}
\mbox{Ad}_{e^{x}} & =e^{\mbox{ad}x}.
\end{align*}
More concretely, suppose that $g=e^{tx}\in G$ and $y\in\mathfrak{g}$.
\begin{align*}
\left.\frac{d}{dt}\mbox{Ad}_{g}\left(y\right)\right|_{t=0} & =\left.\frac{d}{dt}\left(e^{tx}ye^{-tx}\right)\right|_{t=0}=\left[x,y\right]=\mbox{ad}x\left(y\right)
\end{align*}
This is a special case of the relation between any Lie group and Lie
algebra homomorphism
\begin{align*}
\Pi\left(e^{x}\right) & =e^{d\Pi\left(x\right)},
\end{align*}
which is the manifestation of \textbf{Lie's second theorem}: $\mbox{Hom}\left(G,G'\right)=\mbox{Hom}\left(\mathfrak{g},\mathfrak{g}'\right)$
if $G$ is simply connected. The corresponding statement for representations
$\Pi$ is that $\Pi\to d\Pi$ gives the equivalence of the categories
of representations of $G$ and representations of $\mathfrak{g}$.
Moreover, the vector spaces of intertwining operators (morphisms of representations) are isomorphic:
$\mbox{Hom}_{G}\left(V,W\right)=\mbox{Hom}_{\mathfrak{g}}\left(V,W\right)$.


\item \label{Lie subgroup} A \textbf{Lie subalgebra} $\mathfrak{h}$ of $\mathfrak{g}$ is a vector subspace such that $\left[\mathfrak{h},\mathfrak{h}\right]\subset\mathfrak{h}$.
A Lie subalgebra of $\mathfrak{g}$ corresponds to a subgroup of $G$
which is also an immersed submanifold, called a \textbf{Lie subgroup
}of $G$. A map $F:M\to N$ is a \textbf{smooth immersion} if its
differential $dF$ is injective at every point: $\mbox{rank}F=\dim M$.
(To remember the names, an \textbf{i}mmersion is \textbf{i}njective,
whereas a \textbf{s}ubmersion is \textbf{s}urjective.) An \textbf{immersed
submanifold }$S$ of $M$ is a topological manifold (not necessary
having the topology of $M$) together with the inclusion map $S\to M$
which is a smooth immersion.\textbf{ Lie's first theorem }identifies
every Lie subalgebra with a connected Lie subgroup and vice versa.

\item Just be aware that there is no standard definition of a Lie subgroup.
Another notion of a submanifold is that of an \textbf{embedded submanifold},
which can be obtained for instance by setting some coordinates to
zero. The key difference is that embedding has to be a homeomorphism
preserving the global topology. Two classic examples of immersions
that do not preserve the global topology are the ``figure-eight''
map and the irrational winding of a torus. \cite{Lee12} The former maps an open
interval in $\mathbb{R}$ to a closed set in $\mathbb{R}^{2}$. The
latter takes a line with an irrational slope in $\mathbb{R}^{2}$
and map it to a torus. The line will densely wind around the entire
torus. We will call a subgroup which is an embedded submanifold an
\textbf{embedded Lie subgroup}. Fulton and Harris \cite{FH91}, for example, give
the opposite definitions; their Lie subgroups are our embedded Lie
subgroup, whereas our Lie subgroups are their immerse Lie subgroups.

By the \textbf{closed subgroup theorem}, a subgroup is an embedded
Lie subgroup if and only if it is a closed set. Given a closed subgroup
$H\subset G$, this makes $G/H$ a \textbf{homogeneous space}, a smooth
manifold with a transitive group action.

\end{enumerate}

\section{Complex semisimple Lie algebras and their representations}

{\bf From now on, a Lie algebra is over the complex field $k=\mathbb{C}$ unless stated otherwise}. A Lie algebra is semisimple if and only if it has a root decomposition. (\ref{root decomposition}) The tool to study semisimple Lie algebras and their representations is a generalization of the theory of angular momentum in quantum mechanics (\ref{angular momentum}) familiar to every physicist.

\begin{enumerate}

\item \label{glossary} Glossary of notions borrowed from group and ring theories:

An \textbf{ideal} $\mathfrak{h}$ of $\mathfrak{g}$ is a vector subspace
such that $\left[\mathfrak{h},\mathfrak{g}\right]\subset\mathfrak{h}$.
An ideal is automatically a Lie subalgebra. An ideal of $\mathfrak{g}$
corresponds to a \textbf{normal subgroup} $H$ of $G$: $gH=Hg$ for
all $g\in G$.

The \textbf{idealizer }of a set $S\in\mathfrak{g}$ consists of all
elements of $\mathfrak{g}$ that preserve $S$ through the commutator:\textbf{
}$I\left(\mathfrak{h}\right)=\left\{ x|\mbox{ad}x\left(S\right)\subset S\right\} $.
Sometimes the word ``normalizer'' is used instead.

The \textbf{centralizer }of a set $S\in\mathfrak{g}$ consists of
all elements of $\mathfrak{g}$ that commute with all elements of
$S$: $C\left(S\right)=\left\{ x|\mbox{ad}x\left(S\right)=0\right\} $.
The \textbf{center }is the centralizer of the whole Lie algebra. A
commutative Lie algebra is called \textbf{abelian}.

Both idealizers and centralizers are subalgebras.

A Lie algebra $\mathfrak{g}$ is \textbf{solvable }if its \textbf{derived
series }of ideals\textbf{ }$\mathfrak{g}^{i+1}=\left[\mathfrak{g}^{i},\mathfrak{g}^{i}\right]$
terminates. ($\mathfrak{g}^{1}=\mathfrak{g}$.) In an algebraically
closed field, we can always choose a basis such that $\mbox{ad}x$
matrix is upper-triangular for all $x\in\mathfrak{g}$ by the \textbf{Lie
theorem}. Every abelian Lie algebra is solvable.

A Lie algebra $\mathfrak{g}$ is \textbf{nilpotent }if its \textbf{lower
central series }of ideals\textbf{ }$\mathfrak{g}_{i+1}=\left[\mathfrak{g},\mathfrak{g}_{i}\right]$
terminates. ($\mathfrak{g}_{1}=\mathfrak{g}$.) A nilpotent Lie algebra
is automatically solvable. We can always choose a basis such that
$\mbox{ad}x$ matrix is strictly upper-triangular (i.e. nilpotent)
for all $x\in\mathfrak{g}$ by \textbf{Engel theorem}.

\item The standard equivalent definition of a \textbf{semisimple Lie
algebra} is a Lie algebra that has no nonzero solvable ideal. This
implies that semisimple Lie algebras do not have a center because
0 is contained in every Lie subalgebra. Thus, being semisimple is
as far from being abelian as possible. A Lie algebra is \textbf{simple
}if it is not abelian and has no nontrivial ideal. We will see below
that every semisimple Lie algebra is made up of simple Lie algebras.
(If abelian Lie algebras are not excluded, simple Lie algebras may
not be semisimple.) 

With five exceptions, every simple complex Lie
algebra is isomorphic to one of the followings: $\mathfrak{sl}\left(n,\mathbb{C}\right),\mathfrak{so}\left(n,\mathbb{C}\right)$
or $\mathfrak{sp}\left(2n,\mathbb{C}\right)$.

A connection to semisimple (diagonalizable) matrices is that a semisimple
Lie algebra always contain at least one nonzero \textbf{semisimple
element}, $x$ whose $\mbox{ad}x$ matrix is semisimple. This follows
from a generalization of the Jordan decomposition in linear algebra:
every element in a semisimple Lie algebra can be written as a sum
of commuting semisimple and nilpotent element:
\begin{align*}
x & =x_{s}+x_{n},\\
\left[x_{s},x_{n}\right] & =0.
\end{align*}
If $x_{s}=0$ for any $x\in\mathfrak{g}$, then by Engel theorem,
$\mathfrak{g}$ is nilpotent hence solvable, contradicting the semisimplicity
of $\mathfrak{g}$.



\item \label{Levi} \textbf{Levi theorem }states that every Lie algebra can be written
as the sum of its unique largest solvable ideal, the \textbf{radical}
$\mbox{rad}\left(\mathfrak{g}\right)$, and a semisimple Lie subalgebra
$\mathfrak{s}$:
\begin{align*}
\mathfrak{g} & =\mbox{rad}\left(\mathfrak{g}\right)\oplus\mathfrak{s}.
\end{align*}
Recall that solvability is a generalization of abelianity, so every
Lie algebra is a sum of one part that is close to being abelian and
the other part that is far from being abelian. If the first part is
actually abelian, we call the Lie algebra \textbf{reductive}. Any ideal $I$ in a reductive Lie algebra has a complementary ideal $I^{\perp}$ such that $\mathfrak{g} = I \oplus I^{\perp}$.

\item A \textbf{toral subalgebra }is an abelian subalgebra that consists
of semisimple elements. A \textbf{Cartan subalgebra} is a maximal
toral subalgebra which corresponds to a \textbf{maximal torus} in $G$,
hence the name ``toral''.

For semisimple Lie algebras, the standard definition of a Cartan
subalgebra is a toral subalgebra and self-centralizing: $C\left(\mathfrak{h}\right)=\left\{ x|\mbox{ad}x\left(\mathfrak{h}\right)=0\right\} =\mathfrak{h}$.
For the curious reader, the general definition that works for any
Lie algebra is that a subalgebra is Cartan if it is nilpotent and
self-idealizing: $I\left(\mathfrak{h}\right)=\left\{ x|\mbox{ad}x\left(\mathfrak{h}\right)\subset\mathfrak{h}\right\} =\mathfrak{h}$.
(The reduction to centralizing in the semisimple case is, of course,
due to the toricity.) But since we are not going to pursue the general
definition here, we take the semisimple definition to be \emph{the}
definition of a Cartan subalgebra, which can be shown to be equivalent
to the definition as a maximal toral subalgebra in the semisimple
case.

The \textbf{rank }of a Lie algebra is the dimension of its Cartan
subalgebra. Even though a Cartan algebra is not unique, this notion
makes sense because all Cartan subalgebras are conjugated, hence have
the same dimension.

\item \label{Killing form} A bilinear form (or sesquilinear form for a complex vector space)
$\left\langle ,\right\rangle $ is \textbf{invariant }under the action
of $G$ if
\begin{align*}
\left\langle \mbox{Ad}_{g}y,\mbox{Ad}_{g}z\right\rangle  & =\left\langle y,z\right\rangle
\end{align*}
or equivalently, by setting $g=e^{tx}$ and differentiate w.r.t. $t$
at the origin,
\begin{align*}
\left\langle \mbox{ad}x\left(y\right),z\right\rangle +\left\langle y,\mbox{ad}x\left(z\right)\right\rangle  & =0
\end{align*}
i.e. $\text{ad}x$ is antisymmetric w.r.t. the form. Any representation gives an invariant bilinear form $\left\langle x,y\right\rangle =\mbox{Tr}\left(\rho\left(x\right),\rho\left(y\right)\right)$.
It turns out to be very useful to take $\rho$ to be the adjoint representation itself.
Then we have the \textbf{Killing form}
\begin{align*}
\left\langle x,y\right\rangle  & =\mbox{Tr}\left(\mbox{ad}x,\mbox{ad}y\right).
\end{align*}
A bilinear form can be degenerate so it may not be an inner product,
but the Killing form on $\mathfrak{g}$ is nondegenerate if and only
if $\mathfrak{g}$ is semisimple\textbf{ }(\textbf{Cartan's criteria
for semisimplicity}). So it identifies a Cartan subalgebra $\mathfrak{h}$
and its dual $\mathfrak{h}^{*}$.

\item \label{compact Lie algebra} 
A Lie algebra of a compact real Lie group is reductive and has a negative semidefinite Killing form. Conversely, a semisimple real Lie algebra with a negative definite Killing form is a Lie algebra of a compact real Lie group. (There is no nontrivial real Lie algebra with a positive definite Killing form.) A Lie algebra of a compact Lie group is called a {\bf compact Lie algebra}.

The matrices of the adjoint representation can be made antisymmetric i.e. the structure constant can be made totally antisymmetric, if the Lie algebra is a direct sum of simple compact Lie algebras. Example: $\mathfrak{su}(2)$. Counterexample: the adjoint representation of the Heisenberg Lie algebra $\mathfrak{h}$ is nilpotent and thus not antisymmetric. This is consistent with the fact that $\mathfrak{h}$ is not simple because it has a one-dimensional ideal: $[\mathbb{1},x]=0$ for all $x$ in $\mathfrak{h}$.

\item \label{root decomposition} Let $\mathfrak{h}$ be a Cartan subalgebra. It is toral, so $\mbox{ad}h$
for all $h\in\mathfrak{h}$ are simultaneously diagonalizable. Nonzero
eigenvalues of arbitary linear combinations in $\mathfrak{h}$ define
a set (not necessarily a vector space) of \textbf{roots} $\alpha\in R\subset\mathfrak{h}^{*}-\left\{ 0\right\} $.
Each root labels a \textbf{root space} $\mathfrak{h}_{\alpha}=\left\{ x\in\mathfrak{g}|\mbox{ad}h\left(x\right)=\left\langle \alpha,h\right\rangle x\right\} $.
By an abuse of notation, a root can also refer to the number $\alpha\left(h\right):=\left\langle \alpha,h\right\rangle $.
\begin{enumerate}
\item \textbf{Root} \textbf{decomposition}:\textbf{ }the Lie algebra splits
into eigenspaces of $\mbox{ad}h$:
\begin{align*}
\mathfrak{g} & =\mathfrak{h}\oplus\bigoplus_{\alpha\in R}\mathfrak{h}_{\alpha}
\end{align*}

\item $\left[\mathfrak{h}_{\alpha},\mathfrak{h}_{\beta}\right]\subset\mathfrak{h}_{\alpha+\beta}$
\item If $\alpha+\beta\neq0$, then $\mathfrak{h}_{\alpha}$ and $\mathfrak{h}_{\beta}$
are orthogonal w.r.t. the Killing form. This is because
\begin{align*}
\bra{h_{\gamma}} \mbox{ad}h_{\alpha}\mbox{ad}h_{\beta} \ket{h_{\gamma}} & = \braket{h_{\gamma}|h_{\left(\alpha+\beta\right)+\gamma}}=0
\end{align*}
for all $\gamma$. That is, $\mbox{ad}h_{\alpha}\mbox{ad}h_{\beta}$
has no diagonal element. Therefore, the trace is zero.
\end{enumerate}
Roots are special cases of weights. (\ref{weight})

\item A \textbf{root system }is a finite set of vectors (roots) of a
Euclidean space $E$ with rigid geometrical relations:
\begin{enumerate}
\item For any two roots $\alpha$ and $\beta$, the projection of $\beta$
onto $\alpha$ is a half-integer multiple of $\alpha$.
\item The reflection of $\beta$ around the hyperplane $p\cdot\alpha=0$
gives another root. The group generated by these reflections is called
the \textbf{Weyl group}.
\end{enumerate}

Associated with each root $\alpha$ is a \textbf{coroot} $\alpha^{\vee}\in E$
defined by
\begin{align*}
\left\langle \alpha^{\vee},\beta\right\rangle  & =\frac{2\alpha\cdot\beta}{\alpha\cdot\alpha}.
\end{align*}
There is a set of \textbf{simple roots }$\alpha_{j}$ such that every
root $\alpha$ can be uniquely written as an integral linear combination
of simple roots
\begin{align*}
\alpha & =\sum n_{j}\alpha_{j},
\end{align*}
where $n_{j}\in\mathbb{Z}$. Because of the Weyl reflection, we may consider only the set $R_+$ of {\bf positive roots} ($\forall n_{j}>0$).

\item \label{Dynkin diagram} Root systems are in one-one correspondence to semisimple Lie algebras. The point of introducing them is that they can be classified completely according to their {\bf Dynkin diagrams}.

\item \label{weight} We enlarge the root system $R$ to a \textbf{root lattice} $Q$,
an abelian group generated by $R$ (by vector addition in $E$). Similarly,
the \textbf{coroot lattice }$Q^{\vee}$ is generated by $\alpha^{\vee}$.
The \textbf{weight lattice }$P$ is the dual lattice of $Q^{\vee}$:
\begin{align*}
P & =\left\{ \lambda\in E|\left\langle \lambda,\alpha^{\vee}\right\rangle \in\mathbb{Z},\forall\alpha^{\vee}\in Q^{\vee}\right\} .
\end{align*}
(This is simply the definition of a dual lattice.) Let $\pi$ be a
representation of $\mathfrak{g}$ on $V$. A vector $v\in V$ is called
a vector of \textbf{weight} $\lambda\in\mathfrak{h}^{*}$ if $hv=\left\langle \lambda,h\right\rangle v$
for all $h\in\mathfrak{h}$. Then $V$ decomposes into weight spaces:
\begin{align*}
V & =\bigoplus_{\lambda}V_{\lambda}:=\bigoplus_{\lambda}\left\{ v\in V|hv=\left\langle \lambda,h\right\rangle v,\forall h\in\mathfrak{h}\right\} .
\end{align*}
The root lattice is contained in the weight lattice $Q\subset P$
because $\left\langle \alpha^{\vee},\beta\right\rangle \in\mathbb{Z}$.
(They may coincide.) A weight $\lambda$ is {\bf dominant integral} if
\begin{align*}
\langle \lambda,\alpha^{\vee} \rangle \in \mathbb{Z}_+, \forall \alpha \in R_+.
\end{align*}
The {\bf highest weight theorem} then guarantees that each and every dominant integral weight $\lambda$ gives distinct finite dimensional irreps of $\mathfrak{g}$ with the highest weight $\lambda$ and every finite dimensional irrep of $\mathfrak{g}$ arises in this way. A weight is a $\text{rank} \mathfrak{g}$-tuple, so one needs a partial ordering of weights when $\text{rank} \mathfrak{g}>1$ to identify the highest weight.

Weights are also related to the logarithm of the characters of a representation of $G$ restricted to a maximal torus.

\item \label{angular momentum} Let us look at a rank-1 example ($\text{rank}\mathfrak{su}(n) = n-1 $): $\mathfrak{su}(2)$. For concreteness, also let us work in a specific representation.
\begin{align*}
X & =\left(\begin{array}{cc}
0 & 1\\
1 & 0
\end{array}\right),&
Y & =\left(\begin{array}{cc}
0 & -i\\
i & 0
\end{array}\right),&
Z & =\left(\begin{array}{cc}
1 & 0\\
0 & -1
\end{array}\right).
\end{align*}
The Lie algebra $\mathfrak{su}(2)$ of $SU(2)$ is a \emph{real} vector space whose basis can be chosen to be
is $\{ iX,iY,iZ \}$,
\begin{align}
[iX,iY] &= -2iZ,&
[iY,iZ] &= -2iX,&
[iZ,iX] &= -2iY.
\end{align}
The Lie algebra of $\mathfrak{so}(3,\mathbb{R})$ is
\begin{align*}
J_x & =\left(\begin{array}{ccc}
0 & 0 & 0\\
0 & 0 & -1\\
0 & 1 & 0
\end{array}\right),&
J_y & =\left(\begin{array}{ccc}
0 & 0 & 1\\
0 & 0 & 0\\
-1 & 0 & 0
\end{array}\right),&
J_z & =\left(\begin{array}{ccc}
0 & -1 & 0\\
1 & 0 & 0\\
0 & 0 & 0
\end{array}\right),
\end{align*}
\begin{align}
\left[J_x,J_y\right] & =J_z,&
\left[J_y,J_z\right] & =J_x,&
\left[J_z,J_x\right] & =J_y.
\end{align}
These two can be identified via the isomorphism $iX \to -2J_x$, etc. that lifts to a homomorphism between the groups. Physicists' commutation relations of $\mathfrak{su}(2)$ are in fact those of $\mathfrak{su}(2)_{\mathbb{C}}$, the complexified $\mathfrak{su}(2)$.
\begin{align}
\left[X,Y\right] & =2iZ,&
\left[Y,Z\right] & =2iX,&
\left[Z,X\right] & =2iY,
\end{align}
No two of them can be simultaneously diagonalized, so we pick one of
them, say $Z$, to be a basis of the one-dimensional Cartan subalgebra.
The complexification allows us to use the {\bf Chevalley} or {\bf Cartan basis} which incorporates the raising and lowering operators:
\begin{align*}
H & =Z,&
E & =\frac{X+iY}{2} = \left(\begin{array}{cc}
0 & 1\\
0 & 0
\end{array}\right),&
F & =\frac{X-iY}{2} = \left(\begin{array}{cc}
0 & 0\\
1 & 0
\end{array}\right),
\end{align*}
\begin{align*}
\left[H,E\right] & =2E,&
\left[H,F\right] & =-2F,&
\left[E,F\right] & =H.
\end{align*}
isomorphic to $\mathfrak{sl}(2,k)$. When $k=\mathbb{C}$, it is the lowest-dimensional complex semisimple Lie algebra. (In dimensions one and two, we get either an abelian Lie algebra or $gl(2,\mathbb{C})$ which is reductive (\ref{Levi}) but not semisimple.) For finite dimensional representations, representations of $\mathfrak{sl}(2,\mathbb{C})$ corresponds to unitary representations of $su(2)_{\mathbb{C}}$ and hence its real form $\mathfrak{su}(2)=\mathfrak{so}(3,\mathbb{R})$. $SL(2,\mathbb{C})$ is not compact, so it does not have any finite-dimensional unitary irrep and the correspondence is more complicated, but we will not need it in any way.

We know from quantum mechanics that these commutation relations completely determine all $n+1$-dimensional irreps $V_n$ of $SU(2)$, each with the highest weight $n$. $V_n$ splits into the direct sum of all one-dimensional weight spaces
\begin{align*}
V_n &= \bigoplus_k V_k,\\
EV_k &\subset V_{k+2},\\
FV_k &\subset V_{k-2},
\end{align*}
and $V_n$ can be constructed by applying the lowering operator $F$ to the highest weight vector $ \ket{n,k} \subset V_k$:
\begin{align*}
E\ket{n,k} &= 0.
\end{align*}
(A potential point of confusion: some math books define $\ket{n,k}$ to be $k$ steps \emph{below} the highest weight vector i.e. $F^k\ket{j,j}$.)
Irreps can be projected out of a representation as eigenspaces of the {\bf Casimir operator}. Given an orthonormal operator basis $\{|E_{\alpha}) \}$ of $\mathfrak{g}$, the (quadratic) Casimir operator is the identity operator
\begin{align*}
\sum_{\alpha} |E_{\alpha})(E_{\alpha}|,
\end{align*}
where
\begin{align*}
(E_{\alpha}|=|E_{\alpha})^{\dagger}
\end{align*}
For $\mathfrak{sl}(2,\mathbb{C})$, it is
\begin{align*}
X^2+Y^2+Z^2 &= H^2 + EF + FE.
\end{align*}
Note that the Casimir operator does not lie in the Lie algebra since only the Lie bracket is defined there. (It lies in the {\bf universal enveloping algebra}.)

The two- and three-dimensional representations above are $V_1$ and $V_2$ with weights -1,1 and -2,0,2 respectively. $V_1$ is a natural representation of the Lie algebra of $SU(2)$. $V_2$ is the adjoint representation and the natural representation of the Lie algebra of $SO(3,\mathbb{R})$. The simple root 2 generates the root lattice isomorphic to even integers $2\mathbb{Z}$ strictly contained in the weight lattice $\mathbb{Z}$. Representations with the highest weights outside the root lattice are (projective) {\bf spin} or {\bf spinor representations} of $SO(3,\mathbb{R})$.

\end{enumerate}

\section{Compact and Semisimple Lie Groups}

Fundamental groups, homotopy groups...

Non-connected, non-simply connected semisimple Lie groups, maximal compact subgroups, QR decomposition, Gram-Schmidt process as deformation retraction to a maximal compact subgroup, Iwasawa decomposition as a generalized QR decomposition,...

Real forms, compact real form of classical Lie groups, the Lorentz group SO(1,3), conformal groups for $p+q>2$...

Measure and Riemannian metric on compact groups, symmetric spaces...

Hilgert and Neeb could be a really good reference for this section.

\begin{thebibliography}{99}

\bibitem{KirillovJr08}
Alexander Kirillov Jr., \emph{An Introduction to Lie Groups and Lie Algebras}, Cambridge University Press, 2008.

\bibitem{Hall03}
Brian C. Hall, \emph{Lie Groups, Lie Algebras, and Representations: An Elementary Introduction}, Springer, 2003.

%\bibitem{Procesi07}
%Claudio Procesi, \emph{Lie Groups: An Approach through Invariants and Representations}, Springer, 2007.

%\bibitem{Knapp02}
%Anthony W. Knapp, \emph{Lie Groups Beyond an Introduction}, 2nd ed., Birkh{\"a}user, 2002.

\bibitem{Lee12}
John M. Lee, \emph{Introduction to Smooth Manifolds}, 2nd ed., Springer, 2012.

\bibitem{FH91}
William Fulton and Joe Harris, \emph{Representation Theory: A First Course}, Springer, 1991.

%\bibitem{Brif98}
%C. Brif and A. Mann, \emph{Phase-space formulation of quantum mechanics and quantum state reconstruction for physical systems with Lie-group symmetries}, \href{http://dx.doi.org/10.1103/PhysRevA.59.971}{Physical Review A {\bf 59} 971 (1998).}

%\bibitem{Kerr96}
%Megan Kerr, \emph{Some new homogeneous Einstein metrics on symmetric spaces}, \href{http://www.ams.org/journals/tran/1996-348-01/S0002-9947-96-01512-7/S0002-9947-96-01512-7.pdf}{Transactions of the American Mathematical Society {\bf 348} 153 (1996).}

%\bibitem{Geiges08}
%Hansj{\"o}rg Geiges, \emph{An Introduction to Contact Topology}, Cambridge University Press (2008).

%\bibitem{Barut86}
%A. O. Barut and R. Raczka, \emph{Theory of Group Representations and Applications}, 2nd ed., World Scientific, Singapore (1986).

%\bibitem{Oeckl15}
%Robert Oeckl, \emph{Coherent states in the fermionic Fock space}, \href{http://dx.doi.org/10.1088/1751-8113/48/3/035203}{Journal of Physics A: Mathematical and Theoretical {\bf 48} 035203 (2015).}

\end{thebibliography}

\end{document}
